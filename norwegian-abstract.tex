\section*{Sammendrag}

Tradisjonelle databasesystemer møter ikke prestasjonskravene satt av dagens
webapplikasjoner. Kompromisser i form av intrikate mellomlagringshierarkier og
manuell materialisering av data løser deler av hastighetsproblemet, på
bekostning av økt kompleksitet. Soup er et nytt strukturert lagringssystem som
skalerer til millioner av lesninger per sekund på én maskin, uten
overflødigheter observert i andre lagringsoppsett. Ved å propagere oppdateringer
gjennom en dataflytsgraf, hvor forhåndsberegnede resultater opparbeides
inkrementelt ved nodene i grafen, beveger Soup mye av prosesseringsarbeidet fra
lesing til skriving.

Soup lagrer all data i flyktig hovedminne og skriver oppdateringer til en
loggfil for å vedlikeholde data ved fatale feil. Selv om rask
minnelesingshastighet er essensielt for mellomlagrede verdier som aksesseres
ofte, er det et tungvint krav for kjernetabellene, da disse gjerne sjelden
svarer på leseforespørsler. Ved å implementere en diskbasert indeksstruktur over
RocksDB-lagringsmotoren, tar denne avhandlingen Soup fra å være et rent
minnebasert system, til et strukturert lagringssystem kapabelt til å håndtere
mer data enn det har plass til i minne, med kun en minimal nedgang i
skrivegjennomstrømning.

Med kjernetabellene lagret trygt på disk, kan Soup gjenopprettes etter feil ved
gradvis oppbygging av partielt materialiserte resultater. I likhet med
mellomlagringssystemer som starter uten innhold, fører dette til redusert
initiell hastighet. Hastigheten økes sakte, men sikkert, ettersom forespørsler
etter data fyller opp de minnebaserte mellomlagringslokasjonene. Med mål om å
vedlikeholde samme hastighet, implementerer også denne avhandlingen en metode
for å gjennomføre et globalt kontrollpunkt av lokal tilstand i Soup. Ved å se på
dataflytsgrafen som et distribuert system, tar metoden et koordinert
lagringsbilde av data i systemet, slik at Soup kan gjenopprettes på en tiendel
av tiden ved feil.

\newpage
