\documentclass{report}

\usepackage[utf8]{inputenc}
\usepackage{listings}
\usepackage{color}
\usepackage{xcolor}
\usepackage{tikz}

\usetikzlibrary{positioning}
\usetikzlibrary{shapes,decorations,shadows}

\begin{document}
\title{Efficient recovery in a data-flow based storage system}
\author{Martin Ek}
\date{\today}
\maketitle

\newcommand{\note}[1]{{\color{olive}\sffamily #1}}
\newcommand{\todo}[1]{{\color{purple}\sffamily[TODO: #1]}}
\newcommand{\eg}{\textit{e.g.},\xspace}
\newcommand{\ie}{\textit{i.e.},\xspace}
\newcommand{\vs}{\textit{vs.}\xspace}
\newcommand{\viz}{\textit{viz.},\xspace}

% Handy superscript macros
\newcommand{\stss}{\textsuperscript{st}}
\newcommand{\ndss}{\textsuperscript{nd}}
\newcommand{\rdss}{\textsuperscript{rd}}
\newcommand{\thss}{\textsuperscript{th}}

% Enumeration commands
\newcommand{\one}{({\em i}\/)}
\newcommand{\two}{({\em ii}\/)}
\newcommand{\three}{({\em iii}\/)}
\newcommand{\four}{({\em iv}\/)}
\newcommand{\five}{({\em v}\/)}

\newcommand\code[1]{\texttt{#1}}


\begin{abstract}
  Abstract
\end{abstract}

\tableofcontents
\pagebreak

\chapter{Introduction}

\chapter{Background}
\section{Soup}
Modern web applications face an increasingly difficult performance problem in
relation to their storage needs: traditional relational databases perform too
poorly when used in isolation. Either by failing to scale to a large amount of
concurrent users, or by taking too long to return results for more expensive
aggregation queries.

This is often worked around by building up intricate cache hierarchies
\cite{facebook}, similar to the multi-level cache hierarchy
readers might be accustomed to seeing in their processors. This solves both
performance problems: concurrent readers can
access materialized data from the upper levels of the cache hierarchy, achieving
low latency and high throughput even when faced with large amounts of clients.
Expensive queries then only need to be performed once, as long as the result is
invalidated when the underlying data changes. This touches upon the main problem
with these cache hierarchies: invalidation. Write operations still need to change
the underlying storage layer, but are now also responsible for clearing out or
updating the materialized cache levels.

This adds an extra layer of complexity, and is a trade-off accepted by almost
every modern day developer that faces more than a trivial amount of load to
their storage system.

Soup sets out to solve this by providing a single high performant system,
removing the need for manual materialization through cache hierarchies. The
current prototype scales to millions of writes and reads per second
\cite{soup} - both on a single machine and in a distributed fashion.

Soup does this by materializing data automatically in a data-flow
graph, building on existing streaming data-flow research \cite{naiad, dataflow}
by combining it with ideas from performant materialized view solutions
\cite{dbtoaster, pequod}.

\subsection{Data-flow Graph}
Soup turns base table schemas and a pre-defined list of queries into a data-flow
graph that doubles as a set of materialized views. The base tables form the
root nodes of the graph, and all writes propagate from here. The graph itself
resembles the query graph of a traditional relational database system
\cite{codd}, with the graph's intermediate nodes being relational operators.

\section{Rust}

\chapter{Results}
\chapter{Discussion}

\bibliographystyle{acm}
\bibliography{sources}

\end{document}
