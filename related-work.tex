\chapter{Related work}\label{chap:related-work}

\todo{todo!}
\newpage

\section{Indexing}

Index structures are used by databases to facilitate efficient retrieval
queries. While a majority of traditional database systems maintain indices
separate from the data itself (which could be stored in \eg a heap
file~\cite{microsoft-heap, psql-heap}), it has become increasingly common to
co-locate rows with the index---often referred to as a clustered index.
Systems such as InnoDB~\cite{innodb-source} and comdb2~\cite{comdb2} rely
heavily on B-trees for both indexing and row storage, achieving overall decent
read performance on a wide variety of storage mediums.

At the same time, a gradual increase in write-intensive applications have
resulted in a myriad of log-structured merge tree based storage systems---a
data-structure which usually requires less write amplification than
B-trees~\cite{lsm-vs-b}. While LSM-tree based systems (\eg Google
Bigtable~\cite{bigtable} and Apache HBase~\cite{hbase}) provide excellent
availability and scalability, their key-value based APIs are restrictive, and
lack features such as secondary indexing.

Key-value APIs are sufficient for many applications, while others require more
advanced features. Google Spanner~\cite{spanner} and its open-source competitors
remove the need to compromise between strong consistency and scalability, and
provide an SQL-based query interface to its users. While Spanner combines an
implementation based on Bigtable with Paxos to provide distributed consistency,
CockroachDB~\cite{cockroach} and TiDB~\cite{tidb}---both open-source---do the
same with the LSM-tree key-value store RocksDB~\cite{rocksdb} and the Raft
consensus algorithm~\cite{raft}.

Both CockroachDB and TiDB implement advanced features (\eg replication and
sharding) as layered abstractions, with RocksDB's ordered key-value API at the
core. With clever key schemes and heavy use of RocksDB's iteration properties,
CockroachDB and TiDB can support secondary indices on top of RocksDB---a
well-supported and heavily tested library with reliable performance guarantees.
Other projects, such as SLIK~\cite{slik}, HyperDex~\cite{hyperdex}, and
Replex~\cite{replex}, implement secondary indexing as first class citizens in
new distributed key-value stores built from the ground up.

% This section investigates how systems such as CockroachDB and TiDB implement
% secondary index schemes on top of a key-value API, while taking a look at
% existing research in the same field.

\subsection{Secondary indices with LSM-trees}

LSM-tree systems achieve high write throughput in part by buffering updates in
memory, amortizing the disk write penalty across a batch of writes.
AsterixDB~\cite{asterix-storage} recognizes the effectiveness of the LSM-tree
approach, and applies the same technique to in-place update index structures.
The process---which they refer to as LSM-ification---lets AsterixDB build
secondary indices using data structures that generally perform better while
doing so, \eg B-trees. While this is an interesting approach, on-disk data
structures are far from trivial to implement, and additional data structures
undoubtedly increase a system's overall complexity---even if the data structures
are built on the same components.

Another approach is to build index schemes on top of the existing APIs provided
by LSM-trees, notably the \code{Get}, \code{Put}, \code{Delete}, and \code{Seek}
operations provided by systems such as RocksDB.\@ In~\cite{lsm-comparison},
LSM-tree index structures are split into two categories: \textit{standalone}
and \textit{embedded}. The former maintains secondary indices in separate
key-spaces, while the latter embeds the necessary information without
necessitating additional stored rows.

\subsubsection{Standalone Indices}

In traditional relational database management systems, indices are usually
maintained as separate data structures (\eg B-trees) with pointers to primary
key values. Standalone indexing in~\cite{lsm-comparison} stores pointers as
well, either in the same key-space as regular updates or in separate tables
(often referred to as \textit{column families} in LSM-based systems). These
pointers can be maintained either \textit{eagerly} or \textit{lazily}, \eg
either synchronously on the main-path or asynchronously in the background. The
latter introduces complexity to the system, but offers potentially improved
insertion and update performance in return~\cite{deli, pnuts}.

% TODO: list vs composite
% list is not really a valid alternative
% composite depends on the iteration performance, requires careful key schemes
% to make sure that ordering is correct
% potentially a lot of random reads by following pointers, can inline data
% instead. cite pingcap, cockroachdb, myrocks

% embedded, their brief experiments. interesting idea, but requires additional
% modifications to implement top k efficiently.
