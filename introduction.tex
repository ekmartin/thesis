\chapter{Introduction}

Building sophisticated web applications while scaling to potentially millions of
users forces developers to compromise between performance, user requirements,
and application complexity. Whereas traditional relational databases logically
are able to fulfill the increasingly complex storage demands of today's internet
businesses, they are far from able to do so at the scale and performance
required. To continue serving requests at increasing throughputs with low
latencies, developers introduce mitigation strategies ranging from complex cache
hierarchies~\cite{memcached} to denormalized schemas~\cite{denormalization}.

These methods are usually used to drastically improve read performance, while
penalizing write throughput and increasing application complexity.
Soup~\cite{xylem} sets out to solve this dilemma once and for all, with a
structured storage system capable of horizontally scaling to millions of reads
per second, without the need for complex cache deployments or manual maintenance
of materialized views.

Soup achieves this through use of an incrementally maintained data-flow graph.
New updates propagate through the graph at write-time, with pre-computed results
stored at selected \textit{materialized} nodes throughout the graph. This
moves the bulk of the workload from reads to writes, by allowing read operations
to directly access computed state from materialized nodes further down the
graph.

Soup ensures durability by persisting all updates to a write-ahead log before
they are injected into the data-flow graph. While appending entries to a file is
good for performance, recovering from an ever-growing log after a failure is far
from feasible. This thesis improves Soup's durability situation with two main
contributions: it moves Soup's otherwise in-memory table structures to durable
storage, and implements snapshotting of Soup's materialized views. Both
contributions were implemented in the open-source Soup prototype written in the
Rust programming language, and a list of the changes is available in appendix A.

\newpage

\section{From main-memory to durable storage}

After updates are persisted to Soup's write-ahead log, they are injected into
the first nodes in the data-flow graph: the base tables. Unlike the partially
materialized nodes further down the data-flow graph, the base tables can never
be evicted from, and must together always contain a full representation of a
Soup application's state. On the other hand, the base tables should only be
responsible for serving a small part of Soup's read queries. The rest should be
handled by materialized nodes towards the bottom of the graph, using state that
was pre-computed when the updates propagated through the data-flow graph.

This makes volatile main-memory a poor destination for Soup's base table data.
Applications where data is continuously inserted would cause Soup's memory
footprint to grow continuously over time, until eventually reaching its host
system's memory limit. Moving the base tables to durable storage avoids this
problem, while reducing Soup's overall memory usage.

Storing base tables on durable storage also improves Soup's recovery situation,
by avoiding the need to replay the entire write-ahead log after failures. With
all updates safely persisted to and readily available from durable storage,
recovery is instead a matter of replaying data from the base tables when needed.

\section{Snapshotting materialized views}

With durable base tables, Soup recovers significantly faster than by having to
replay the entire write-ahead log. This is not without downside however: whereas
log-based recovery brings all nodes in the data-flow graph back to a
pre-failure state, durable base tables leave partial nodes empty, resulting in a
latency penalty for initial read-queries. Instead, we would like to periodically
write \textit{snapshots} of the materialized state at each node to durable
storage, ensuring a speedy recovery process for both base tables and
materialized views alike.

To snapshot nodes individually while maintaining consistency, it is crucial that
all nodes snapshot the same window of updates. For a given update at any given
time, said update must either be contained in every snapshot across the graph,
or neither of them. While updates are processed synchronously within a single
domain in Soup (a partition of nodes), data flows asynchronously between
domains, where the boundaries can be both within a local machine and across a
network. At the same time, taking a snapshot should not incur a significant
pause in processing, which would result in lower throughput all around.

By approaching the problem from the viewpoint of snapshotting in a distributed
system, this thesis implements a snapshotting method capable of creating a
logically consistent snapshot across the data-flow graph, with a focus on
maintaining as much of Soup's performance guarantees as possible.

\section{Outline}

The rest of this thesis is structured as follows:

\begin{itemize}
  \item \textbf{Chapter~\ref{chap:background}} introduces core theory behind fundamental
  concepts used in the rest of this thesis.
  \item \textbf{Chapter~\ref{chap:related-work}} reviews ideas from research and
  industry relevant to the thesis' main contributions.
  \item \textbf{Chapter~\ref{chap:benchmarks}} describes new and existing
  benchmarks used throughout the thesis.
  \item \textbf{Chapter~\ref{chap:persistent-bases}} outlines the requirements
  for a persistent base table implementation, followed by two implementation
  iterations.
  \item \textbf{Chapter~\ref{chap:recovery}} gradually builds up a snapshotting
  implementation.
  \item \textbf{Chapter~\ref{chap:evaluation}} evaluates the resulting
  implementations from the two previous chapters, using the benchmarks
  introduced in chapter~\ref{chap:benchmarks}.
  \item \textbf{Chapter~\ref{chap:conclusion}} presents possible next steps
  towards a production-ready Soup, while concluding on the results presented
  earlier in the thesis.
\end{itemize}
