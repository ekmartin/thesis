Traditional database systems do not meet the throughput demands of today's web
applications. Mitigation strategies on the form of intricate cache hierarchies
and manual view materialization solve parts of the performance equation, at the
cost of increasing complexity. Soup is a new structured storage system that
scales to millions of reads per second on a single machine, without the
architectural complexity seen in other storage deployments. By propagating
updates through a data-flow graph, where pre-computed state is incrementally
maintained at materialized nodes throughout the graph, Soup moves the majority
of the processing work from reads to writes.

Soup stores all data in volatile main-memory and relies on a write-ahead log for
durability. While fast memory access is crucial for frequently accessed
materialized views, it is a cumbersome requirement for Soup's base tables, which
only serve read requests at rare occasions. By implementing a disk-resident
index structure on top of the RocksDB storage engine, this thesis moves Soup
from a pure main-memory database, to a structured storage system capable of
handling datasets larger than its memory size, with only a small decrease in
overall write throughput.

With its base tables stored safely on durable storage, Soup can recover from
fatal failures by gradually building up its partially materialized views as
needed. Similar to a system that recovers with an empty cache, this reduces
initial performance while live requests slowly build up Soup's memory-based
state. To completely remove the performance degradation of recovery, this thesis
also implements a method for performing a global checkpoint of Soup's
materialized state. By approaching the data-flow graph as a distributed system,
the method performs a coordinated snapshot of all local state, allowing Soup to
recover in about a tenth of the time.
